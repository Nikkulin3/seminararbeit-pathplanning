%   Version 3.0 | Lehrstuhl Produktionsinformatik  |  27.08.2020
%%%%%%%%%  Zusätzlich benötigte Pakete in den Header einfügen  %%%%%%%%%%
\input{src/header.tex}

%%%%%%%%%%%%%    Einbinden der Quellen    %%%%%%%%%%%%%%%%%%%%%%%%%%%%%%%
\addbibresource{src/references.bib}

%%%%%%%%%%%%%%%   Folgenden Abschnitt anpassen!    %%%%%%%%%%%%%%%%%%%%%%
% Arbeit
\newcommand{\artderausarbeitung}{Seminararbeit}
\newcommand{\titelderarbeit}{Verbesserung der Pfad- und Trajektorienplanung am UR5}
% Autor
\newcommand{\namedesautors}{Nik Julin Nowoczyn}
\newcommand{\matrikelnummer}{8529776}
\newcommand{\studiengang}{Ingenieurinformatik}
\newcommand{\pruefer}{Prof.~Dr.-Ing.~Johannes Schilp}
\newcommand{\betreuer}{Ludwig Vogt}
\newcommand{\ausgabedatum}{01.\,10.\,2022}
\newcommand{\abgabedatum}{31.\,03.\,2023}
%%%%%%%%%%%%%%%%%%%%%%%%%%%%%%%%%%%%%%%%%%%%%%%%%%%%%%%%%%%%%%%%%%%%%%%%%
% Hier können eigene Befehle angelegt werden:
%\newcommand{\Name}{Definition}

%%%%%%%%%%%%%%%%%%%%%%%%%%%%%%%%%%%%%%%%%%%%%%%%%%%%%%%%%%%%%%%%%%%%%%%%%
\begin{document}
    \onehalfspacing
    % Wird auf alle nicht leeren Seiten angewandt
% siehe auch;
% http://tug.ctan.org/tex-archive/macros/latex/contrib/fancyhdr/fancyhdr.pdf
\fancypagestyle{title}{%
	\fancyhf{}
	\fancyhead[L]{Fakultät für Angewandte Informatik \\
		\textbf{Lehrstuhl für Produktionsinformatik} \\ Prof. Dr.-Ing. Johannes Schilp}

	\fancyhead[R]{
		\raisebox{0pt}[0pt][0pt]{\includegraphics[width=3.5cm]{src/Uni_Aug_Logo_Basis_pos_A.pdf}}
	}

	\renewcommand{\headrulewidth}{0.4pt} %obere Trennlinie
}


\renewcommand{\chaptermark}[1]{\markboth{
		\thechapter. #1}{}}
%\renewcommand{\sectionmark}[1]{\markboth{
%		\thesection. #1}{}}


\pagestyle{fancyplain}% <- pagestyle fancyplain
\renewcommand\plainheadrulewidth{.4pt}% headrule on plain pages
\fancyhf{}
\fancyhead[L]{\namedesautors}
\fancyhead[R]{\leftmark}
%\fancyhead[R]{\rightmark}
\fancyfoot[RE,LO]{}
\fancyfoot[LE,RO]{\pagemark}
\renewcommand{\chaptermark}[1]{%
	\markboth{#1}{}}


\headsep = 20pt % COOLER CODE: Das Ding setzt die Margin zwischen Header und Text darunter
				% Bei Überschriften setzt sich diese Margin aber aus dieser bottom Margin + der
				% top Margin von Überschriften zusammen. Diese kann mit dem folgenden Command
				% geändert werden:
\renewcommand*{\chapterheadstartvskip}{\vspace*{0cm}} % COOLER CODE:
													  % Setzt top Margin bei \chapter. Wenn dus
													  % so haben willst wie vorher, einfach die
													  % Zeile löschen. Selbe gilt bei \headsep

%%  Der Vollständigkeithalber die alten verschiedenen Pagestyles von Peter
%   pagestyle.tex
%   Version 1.0     |   Peter Krönes    |   08.05.2018

%%%%%%%%%%%%%%%%%%%%%%%  PAGESTYLE  %%%%%%%%%%%%%%%%%%%%%%%%%%%%%%%%%%%%
%
%\usepackage{fancyhdr}
%\pagestyle{fancy}
%\renewcommand{\chaptermark}[1]{\markboth{
%		\thechapter. #1}{}}
%\renewcommand{\sectionmark}[1]{\markboth{
%		\thesection. #1}{}}
%
%%%%%%%%%%%%%%%%%%%%%%%%%%%%%%%%%%%%%%%%%%%%%%%%%%%%%%%%%%%%%%%%%%%%%%%%
%\fancypagestyle{scrheadings}{
%	\fancyhead[LE,RO]{\nouppercase\leftmark} 	% Kapitelname oben ins andere Eck
%	\fancyhead[LO,RE]{\namedesautors}	% Autor oben im Eck
%	\fancyfoot[LE,RO]{\pagemark} 		% Seitenzahl unten im Eck
%	\fancyfoot[C]{}
%}
%%%%%%%%%%%%%%%%%%%%%%%%%%%%%%%%%%%%%%%%%%%%%%%%%%%%%%%%%%%%%%%%%%%%%%%%
% Pagestyle für die erste Seite der List of Figures/Tables
%\fancypagestyle{pagenumberstyle}{
%	\renewcommand{\headrulewidth}{0pt}
%
%	\fancyhead[LE,RO]{\chaptermark} 	% Kapitelname oben ins andere Eck
%	\fancyhead[LO,RE]{\namedesautors}	% Autor oben im Eck
%	\fancyfoot[LE,RO]{\pagemark} 		% Seitenzahl unten im Eck
%	\fancyfoot[C]{}
%}
% Unterscheidung zwischen einseitig und zweiseitigem Druck nicht nötig, da Kapitel bei zweiseitigem Druck immer rechts anfangen.
%%%%%%%%%%%%%%%%%%%%%%%%%%%%%%%%%%%%%%%%%%%%%%%%%%%%%%%%%%%%%%%%%%%%%%%%
%
%%%%%%%%%%%%%%%%%%%%%%%%%%%%%%%%%%%%%%%%%%%%%%%%%%%%%%%%%%%%%%%%%%%%%%%%
% Pagestyle für scrartcl
%\fancypagestyle{articlestyle}{
%	\renewcommand{\headrulewidth}{0.1pt}
%	\fancyhead[LE,RO]{\chaptermark} 	% Kapitelname oben ins andere Eck
%	\fancyhead[LO,RE]{\namedesautors}	% Autor oben im Eck
%	\fancyfoot[LE,RO]{\pagemark} 		% Seitenzahl unten im Eck
%	\fancyfoot[C]{}
%}
% Unterscheidung zwischen einseitig und zweiseitigem Druck nicht nötig, da Kapitel bei zweiseitigem Druck immer rechts anfangen.
%%%%%%%%%%%%%%%%%%%%%%%%%%%%%%%%%%%%%%%%%%%%%%%%%%%%%%%%%%%%%%%%%%%%%%%%



%% wurde dann vor listoffigures, listoftables, tableofcontents und printbibliography
%% wie folgt eingestellt: \tableofcontents{\thispagestyle{pagenumberstyle}}

    %   titlepage.tex

\begin{titlepage}
\thispagestyle{title}


\vspace*{65mm}

\centering
{\Huge \textbf{\artderausarbeitung}}\\[5ex]
{\Large \textbf{\titelderarbeit}}\\[5ex]

\vspace{35mm}

\flushleft
\begin{tabular}{ll}
Vorgelegt von:          & \quad \namedesautors \\[1,5ex]
Matrikelnummer:			& \quad \matrikelnummer \\[1,5 ex]
Studiengang:            & \quad \studiengang \\[1,5ex]
Prüfer:					& \quad \pruefer \\[1,5ex]
Betreuer:				& \quad \betreuer \\[1,5ex]
Ausgabedatum:           & \quad \ausgabedatum \\[1,5ex]
Abgabedatum:            & \quad \abgabedatum\\[1,5ex]
\end{tabular}
\vfill
\end{titlepage}
\cleardoublepage

    \pagenumbering{Roman}
    \onehalfspacing
    \chapter*{Aufgabenstellung}
\addcontentsline{toc}{chapter}{Aufgabenstellung}
\chaptermark{Aufgabenstellung}

In der Industrie der heutigen Zeit werden zur Herstellung verschiedenster Produkte Industrieroboter eingesetzt.
Die Aufgabe der Forschung ist es hierbei, die Fähigkeiten dieser Roboter auf neue Produktionsverfahren zu übertragen.
In der Universität wird der Roboter UR5 von Universal Robotics unter anderem verwendet, um automatisierte additive Fertigung zu ermöglichen.
Dabei sollen die gefertigten Bauteile automatisiert aus dem Druckbereich entfernt und an anderer Stelle platziert werden.

In dieser Arbeit soll die Steuerung des UR5 näher untersucht werden, die einige Probleme aufweist.
Es wurde unter anderem festgestellt, dass der Roboter beim Fahren zwischen zwei Positionen oftmals in eine Position fährt, in der es nicht mehr weitergeht und in Folge ein Notstopp durchgeführt werden muss.
Es gilt, eine mögliche Ursache für dieses Problem zu identifizieren und eine Lösung zu präsentieren, die es ermöglicht Kollisionen frühzeitig zu erkennen und bei möglichst gleichem Pfad durch andere Bewegungsmuster zu umgehen.
    \chapter*{Kurzfassung}
\addcontentsline{toc}{chapter}{Kurzfassung}
\chaptermark{Kurzfassung}


In dieser Arbeit wird die Steuerung des UR5 näher untersucht und eine Ursache der häufig auftretenden Notstopps ermittelt.
Es stellt sich heraus, dass die Ursache für die Stopps oftmals ein unmöglicher vom Benutzer vorgegebener Bewegungsablauf ist, der aber vom Roboter erst kurz vor einer möglichen Kollision registriert wird.
Infolge wird dann immer ein Notstopp ausgeführt, der einen manuellen Neustart des Roboters erforderlich macht.
Als Lösung des Problems werden zwei Lösungsmethoden vorgestellt und in einer Simulationsumgebung implementiert:
Die Berechnungsmethode zur Bestimmung des kürzesten Weges kann verwendet werden, falls eine lineare Bewegung des Endeffektors zwischen Start- und Zielpunkt möglich ist,
andernfalls wird auf eine Methode zur Bestimmung der direkten Bewegung zurückgegriffen.

Um die obigen Algorithmen zu unterstützen, wird außerdem eine Methode zur Kollisionsüberprüfung und eine geometrische Berechnungsmethode zur Bestimmung der inversen Kinematik vorgestellt,
die für jede Endeffektor-Transformation in der Regel bis zu acht mögliche Roboterstellungen berechnen kann.

Um die implementierten Lösungen zu beschreiben, wird zunächst in Kapitel~\ref{ch:direkte-kinematik} auf die direkte Kinematik und die DH-Konvention eingegangen, dann in Kapitel~\ref{ch:inverse-kinematik} verschiedene Lösungen für die inverse Kinematik vorgestellt und zum Schluss in Kapitel~\ref{ch:pfadplanung} die Pfadplanungsalgorithmen vorgestellt.
    \tableofcontents        % Inhaltsverzeichnis
    \listoffigures        % Abb.verzeichnis
    \listoftables        % Tabellenverzeichnis
    \chapter*{Abkürzungsverzeichnis}

\begin{acronym}
 \acro{av}[AV]{Autonomes Fahrzeug}
\end{acronym}

% see usage here: https://ramibaddour.com/2017/01/18/latex-working-with-acronyms/
%%%%%%%%%%%%%%%%%%%%%%%%%%%%%%%%%%%%%%%%%%%%%%%%%%%%%%
% zum Ausgeben der Verzeichnisse 
%\symbolverzeichnis

    \cleardoublepage
    \pagenumbering{arabic}
%%%%%%%%%%%%%%     Eigene Kapitel hier einfügen    %%%%%%%%%%%%%%%%%%%%%%
    \cleardoublepage
\chapter{Räumliche Beschreibung}

ggf kürzen und/oder am Ende ergänzen ??

\section{Koordinatensysteme}

\section{Translation}

\section{Rotation}

Euler / current frame,
Fixed Frame,
Axis-Angle,
Quaternion

\section{Homogene Transformationsmatrix}
    \cleardoublepage


\chapter{Direkte Kinematik}\label{ch:direkte-kinematik}

?? Quelle

Die direkte Kinematik ist dafür verantwortlich aus den verschiedenen Winkeln und Positionen der Gelenke die Rotation und Position des Endeffektors im Raum zu berechnen.
Dazu wird zunächst in jedem Gelenk ein Koordinatenursprung gelegt, der eine Nullstellung jedes Gelenks beschreibt.
Um alle Gelenke in einer kinematischen Kette abzubilden, kann dann beispielsweise mithilfe der homogenen Transformationsmatrix (Abschnitt~\ref{sec:homogene-transformationsmatrix}) und einem Parameter in den Freiheitsgraden des entsprechenden Gelenks eine Rechenvorschrift aufgebaut werden, um den Roboter zu beschreiben und die Position des Endeffektors schnell bestimmen zu können.


\section{DH-Konvention}\label{sec:dh-konvention}

?? Quelle

?? Unterschied UR5 und UR5e https://www.universal-robots.com/articles/ur/application-installation/dh-parameters-for-calculations-of-kinematics-and-dynamics/ ??

Die Konvention, die in der Regel verwendet wird, um Rotation und Translation eines Gliedes der kinematischen Kette darzustellen ist die sog.\ \ac{dhk} oder DH-Transformation.
\ac{dhk} beschreibt, wie die Koordinatensysteme basieren auf dem vorherigen Koordinatensystem beschrieben werden.
Um nun ausgehend von Gelenk $n-1$ das Koordinatensystem von Gelenk $n$ zu beschreiben, müssen die folgenden Regeln befolgt werden:

\begin{enumerate}
    \item Achse $z_{n}$ liegt entlang der Bewegungsachse des Gelenks $n$
    \item Achse $x_{n}$ liegt auf der kürzesten Verbindung zwischen Achsen $z_{n-1}$ und $z_{n}$.
    \item Die $y_{n}$ Achse wird rechtshändig ergänzt.
\end{enumerate}

Dabei sind die Ursprünge der Gelenkkordinatensysteme oftmals nicht im Gelenkursprung, was Komplexität für die Berechnung von Transformationen verringert.
Aus der Beziehung der zwei Koordinatensysteme können die DH-Parameter abgeleitet werden (siehe auch Abbildung~\ref{fig:dh-konvention1}):

\begin{itemize}
    \item $\theta_n$ Winkel zwischen $x_{n-1}$ und $x_n$ mit Rotationsachse $z_{n-1}$
    \item $d_n$: Kleinster Abstand zwischen $x_{n-1}$ und $x_n$
    \item $a_n$: Abstand zwischen den Achsen $z_{n-1}$ und $z_n$
    \item $\alpha_n$ Winkel zwischen $z_{n-1}$ und $z_n$ mit Rotationsachse $x_{n}$
\end{itemize}

Dies entspricht den folgenden Transformationsmatrizen (Gleichungen~\ref{eq:dh1},~\ref{eq:dh2},~\ref{eq:dh3},~\ref{eq:dh4}):

\newcommand{\ct}{\cos(\theta_n)}
\newcommand{\st}{\sin(\theta_n)}
\newcommand{\ca}{\cos(\alpha_n)}
\newcommand{\sa}{\sin(\alpha_n)}

\begin{equation}
    T_{\theta_n} =
    \begin{bmatrix}
        \ct & -\st & 0 & 0 \\
        \st & \ct  & 0 & 0 \\
        0   & 0    & 1 & 0 \\
        0   & 0    & 0 & 1 \\
    \end{bmatrix}
    \label{eq:dh1}
\end{equation}
\begin{equation}
    T_{d_n} =
    \begin{bmatrix}
        1 & 0 & 0 & 0   \\
        0 & 1 & 0 & 0   \\
        0 & 0 & 1 & d_n \\
        0 & 0 & 0 & 1   \\
    \end{bmatrix}
    \label{eq:dh2}
\end{equation}

\begin{equation}
    T_{a_n} =
    \begin{bmatrix}
        1 & 0 & 0 & a_n \\
        0 & 1 & 0 & 0   \\
        0 & 0 & 1 & 0   \\
        0 & 0 & 0 & 1   \\
    \end{bmatrix}
    \label{eq:dh3}
\end{equation}

\begin{equation}
    T_{\alpha_n} =
    \begin{bmatrix}
        1 & 0   & 0    & 0 \\
        0 & \ca & -\sa & 0 \\
        0 & \sa & \ca  & 0 \\
        0 & 0   & 0    & 1 \\
    \end{bmatrix}
    \label{eq:dh4}
\end{equation}

Um nun Koordinatensystem $n-1$ in Koordinatensystem $n$ zu überführen, kann die Transformationsmatrix $T_{n-1,n}$ verwendet werden (Gleichung~\ref{eq:dh5}).
Für den beweglichen Freiheitsgrad der Gelenke kann für die einfache Berechnung noch eine Substitution für Variable $q_n$ durchgeführt werden (Gleichung~\ref{eq:subst1} für translatorische und Gleichung~\ref{eq:subst2} für rotatorische Gelenke).

\begin{equation}
    T_{n-1,n}(q_n) \coloneqq T_{\theta_n} \cdot T_{d_n} \cdot T_{a_n} \cdot T_{\alpha_n} =
    \begin{bmatrix}
        \ct & -\st\ca & \st\sa  & a_n\ct \\
        \st & \ct\ca  & -\ct\sa & a_n\st \\
        0   & \sa     & \ca     & d_n    \\
        0   & 0       & 0       & 1      \\
    \end{bmatrix}
    \label{eq:dh5}
\end{equation}

?? Modifizierte dh-parameter, andere Matrix (siehe andersen~\cite{rasmusandersenKinematicsUR52018})

\begin{figure}[h]
    \centering
    \includegraphics[width = .5\textwidth]{Bilder/Denavit-Hartenberg-Transformation.svg}
    \caption{\ac{dhk} zwischen zwei Gelenken~\cite{jahobrCoordinateSystemsDenavitHartenberg2007}}\label{fig:dh-konvention1}
\end{figure}


\begin{equation}
    d_n \coloneqq d_{n_0} + q_n    \label{eq:subst1}
\end{equation}
\begin{equation}
    \theta_n \coloneqq \theta_{n_0} + q_n    \label{eq:subst2}
\end{equation}


\section{Unified Robot Description Format}\label{sec:urdf}

?? Quelle
\ac{urdf} ist ein Standard entwickelt für \ac{ros}, um sowohl die geometrischen als auch physische und visuelle Eigenschaften eines Roboters zu beschreiben.
Für die Beschreibung ist eine auf XML basierende Datei nötig, die mit Tags geschachtelt den Roboter mithilfe von sog. \enquote{links} und \enquote{joints} beschreibt.

Links beschreiben die physischen Verbindungen zwischen zwei Gelenken.
Dabei wird unterschieden zwischen visuellen Eigenschaften (\enquote{visual}), Kollisionseigenschaften (\enquote{collision}) und Trägheitseigenschaften (\enquote{inertial}).
Jede dieser Kategorien kann so die Geometrie des Links aus verschiedenen Perspektiven unterschiedlich beschreiben

Joints bezeichnen Gelenke, also die Verbindungen zweier Links und beschreiben die mögliche relative Bewegung zwischen diesen.
Dabei sind nicht nur translatorische (\enquote{prismatic}) und rotatorische Gelenke (\enquote{revolute}) beschreibbar, sondern auch feste (\enquote{fixed}), schwebende (\enquote{floating}) und planare Verbindungen (\enquote{planar}).
Für die Beschreibung eines Joints wird der vorhergehende Link als \enquote{parent} und der nächste Link als \enquote{child} bezeichnet.
Zudem müssen im Feld \enquote{origin} Translation und Rotation des Ursprungs im Koordinatensystem des Parent Links, sowie im Feld \enquote{axis} je nach Gelenk die Rotationsachse, Translationsachse oder Normale der Bewegungsoberfläche angegeben werden.
Desweiteren ist es möglich, Schnittstellen für die Bewegung der Motoren zu definieren und Bewegungslimits für die Gelenke anzugeben, um die Ansteuerung des Roboters und die Bewegungsplanung zu vereinfachen.


\section{Beschreibung des UR5e}\label{sec:ur5-in-dh}
?? Quelle https://www.universal-robots.com/articles/ur/application-installation/dh-parameters-for-calculations-of-kinematics-and-dynamics/

Der in dieser Arbeit betrachtete Roboter ist der UR5e von Universal Robots.
Dieser ist ein vergleichsweise günstiger Roboter mit verringerter Zahl an Singularitäten (siehe Abschnitt~\ref{sec:singularitaten}), sechs Gelenken und einer offenen kinematischen Kette.
Offiziell wird der Roboter mit den DH-Parametern aus Tabelle~\ref{tab:ur5-dh1} und dynamischen Eigenschaften der Links aus Tabelle~\ref{tab:ur5-dh2} beschrieben.
Abbildung~\ref{fig:ur5-axis} kann zudem die Dimensionierung des Roboters und die Anordnung der Achsen in Nullstellung entnommen werden.

\begin{figure}[h]
    \centering
    \includegraphics[width = .9\textwidth]{Bilder/ur5-axis}
    \caption{Achsen des UR5-Roboters mit Angabe der DH-Parameter (links) und Visualisierung der Nullposition (rechts)~\cite{rasmusandersenKinematicsUR52018}}\label{fig:ur5-axis}
\end{figure}

\begin{table}
    \centering
    \begin{tabular}{lrrrllrl}
        \toprule
        \textbf{Joint} & $\boldsymbol{\theta}$ \textbf{[rad]} & $\boldsymbol{a}$ \textbf{[m]} & $\boldsymbol{d}$ \textbf{[m]} & $\boldsymbol{\alpha}$ \textbf{[rad]}  \\
        \midrule
        Joint 1        & 0                                    & 0                             & 0.1625                        & \pi/2                                \\
        Joint 2        & 0                                    & -0.425                        & 0                             & 0                                    \\
        Joint 3        & 0                                    & -0.3922                       & 0                             & 0                                    \\
        Joint 4        & 0                                    & 0                             & 0.1333                        & \pi/2                                \\
        Joint 5        & 0                                    & 0                             & 0.0997                        & -\pi/2                               \\
        Joint 6        & 0                                    & 0                             & 0.0996                        & 0                                    \\
        \bottomrule
    \end{tabular}
    \caption{DH-Parameter des UR5e Roboters von Universal Robots~\cite{UniversalRobotsDH}}
    \label{tab:ur5-dh1}
\end{table}
\begin{table}
    \centering
    \begin{tabular}{lrrrllrl}
        \toprule
        \textbf{Link} & \textbf{Masse [kg]} & \textbf{Schwerpunkt [m]} \\
        \midrule
        Link 1        & 3.761               & [0, -0.02561, 0.00193]   \\
        Link 2        & 8.058               & [0.2125, 0, 0.11336]     \\
        Link 3        & 2.846               & [0.15, 0.0, 0.0265]      \\
        Link 4        & 1.37                & [0, -0.0018, 0.01634]    \\
        Link 5        & 1.3                 & [0, 0.0018,0.01634]      \\
        Link 6        & 0.365               & [0, 0, -0.001159]        \\
        \bottomrule
    \end{tabular}
    \caption{Beschreibung der dynamischen Eigenschaften des UR5e Roboters von Universal Robots~\cite{UniversalRobotsDH}}
    \label{tab:ur5-dh2}
\end{table}
    \cleardoublepage


\chapter{Inverse Kinematik}

In der direkte Kinematik wird aus den Gelenkzuständen Position und Rotation des Endeffektors bestimmt.
Der umgekehrte Vorgang, also das Berechnen der Gelenkpositionen bei gegebener Zieltransformation, wird als inverse Kinematik bezeichnet.
Dies ist erforderlich, um Aufgaben in der Umgebung des Roboters zu lösen und die angestrebten Zielpunkte zu erreichen.
Dabei muss beachtet werden, dass oft mehrere gleichwertige Lösungen für eine Zielstellung des Endeffektors vorliegen.
Allerdings kann die inverse Kinematik offener Ketten oft nicht einfach direkt mithilfe einer mathematischen Formel gelöst werden (siehe Abschnitt~\ref{sec:analytische-losung}).
Im Fall des UR5, ist dies allerdings in den meisten Fällen möglich.

Die Lösung des Problems kann geometrisch, analytisch oder numerisch gelöst werden.
Auf die ersten beiden Lösungsmöglichkeiten wird im Folgenden näher eingegangen.


\section{Analytische Lösung}\label{sec:analytische-losung}

Für eine direkte Lösung muss die Formel aus Gleichungen~\ref{eq:dh5} und~\ref{eq:subst2} herangezogen werden.
Nach Multiplikation der Transformationsmatrizen müsste die entstehende Gleichung~\ref{eq:inv-kin1} nach den Variablen $q_1$ bis $q_n$ aufgelöst werden.
Da aufgrund der rotatorischen Gelenke nichtlineare trigonometrische Funktionen verwendet werden, ist diese Funktion bei Industrierobotern meist nichtlinear und in vielen Fällen auch nicht lösbar.

\begin{equation}
    T_{0,n}(\overrightarrow{q}) = \prod_{i=1}^{n} T_{i-1,i}(q_i)     \label{eq:inv-kin1}
\end{equation}


\section{Geometrische Lösung}\label{sec:geometrische-losung}
Um die Berechnung zu vereinfachen, wird oftmals in den äußeren drei Gelenken eines Sechsarmroboters ein Handwurzelpunkt eingeführt, in dem sich die Achsgeraden dieser Gelenke schneiden.
Da die Drehung des sogenannten Handgelenks die Position des Handwurzelpunktes nicht verändert, kann so zuerst die Berechnung der Position des Handwurzelpunkts mithilfe der Zieltransformation und im Anschluss unabhängig die Berechnung der restlichen Gelenke stattfinden.
Dieser Trick ist allerdings im UR5 nicht verwendbar, da hier wohl aufgrund der größeren Zahl an Singularitäten dieser Technik (siehe Abschnitt~\ref{sec:singularitaten}) auf die Verwendung eines Handwurzelpunkts verzichtet wurde.

Dennoch kann durch eine Analyse der geometrischen Eigenschaften und der geringen Zahl der Singularitäten des UR5 eine Lösung von $\overrightarrow{q}$ bzw. $\overrightarrow{\theta}$ gefunden werden.
Die Strategie hierbei ist, beim ersten Gelenk zu beginnen und nach und nach die anderen Gelenke mit der Transformation $T_{0,6}$ des Endeffektors zu verknüpfen.
Bekannt sind zu Beginn der Rechnung nur die DH-Parameter $(\theta_i, a_i, d_i, \alpha_i)$ jedes Gelenks $i$ im Ausgangszustand, wobei $\theta_i$ als freie Variable jedes Gelenks betrachtet wird (Abschnitt~\ref{sec:ur5-in-dh}), sowie die Transformation des Endeffektors $T_{0,6}$.
Eine detailliertere Rechnung kann~\cite{rasmusandersenKinematicsUR52018} und~\cite{hawkinsAnalyticInverseKinematics2013} entnommen werden.

\subsubsection{1. Gelenk eins (Basis)}

Zunächst wird der Ausgangspunkt $P_{0,5}$ von Gelenk 5 berechnet (Gleichung~\ref{eq:inv1-1}~\cite[4]{rasmusandersenKinematicsUR52018}) und im Anschluss trigonometrisch Winkel $\theta_1$ bestimmt (Gleichung~\ref{eq:inv1-2} sowie Abbildung~\ref{fig:inv1-1}).
Dabei enstehen zwei Lösungen, die die Schulter des Roboters entweder links oder Rechts vom Ursprung platzieren.

\begin{equation}
    P_{0,5} = T_{0,6} \cdot
    \begin{bmatrix}
        0 \\ 0 \\ -d_6 \\ 0
    \end{bmatrix}
    \label{eq:inv1-1}
\end{equation}
\begin{equation}
    \theta_1 = \arctantwo(P_{0,5y}, P_{0}) \pm \arccos \left( \frac{d_4}{ \sqrt{ P_{0,5x}^2 + P_{0,5y}^2 }  } \right) + \frac{\pi}{2}
    \label{eq:inv1-2}
\end{equation}
\begin{figure}[h]
    \centering
    \includegraphics[width = .5\textwidth]{Bilder/inv1}
    \caption{Berechnung von $\theta_1$, Betrachtung von Gelenk eins bis fünf~\cite{rasmusandersenKinematicsUR52018}}\label{fig:inv1-1}
\end{figure}

\subsubsection{2. Gelenk fünf (oberes Handgelenk)}

Im Anschluss kann $\theta_5$ bestimmt werden, da der y-Teil der Position von Gelenk 6 relativ zu Gelenk 1 ($P_{1,6y}$) nur mithilfe von $\theta_5$ und bereits bekannter Parameter $P_{0,6}$, $\theta_1$ und der DH-Parameter berechnet werden kann (siehe Abbildung~\ref{fig:inv1-2}).
Aus dieser Erkenntnis ergibt sich Gleichung~\ref{eq:inv2-1}
$P_{1,6y}$ erhält man durch die Rotation des Ursprungssystems $T_{0,6}$ um die $z_1$-Achse (Gleichung~\ref{eq:inv2-2}).
In Kombination erhält man die Gleichung für $\theta_5$ (Gleichung~\ref{eq:inv2-3}).
Dabei entstehen wiederum zwei Lösungen, die jeweils das Handgelenk ober- oder unterhalb des Arms platzieren.
\begin{equation}
    - P_{1,6y} = d_4 + d_6 \cdot \cos(\theta_5)
    \label{eq:inv2-1}
\end{equation}
\begin{equation}
    P_{1,6y} = - P_{0,6x} \cdot \sin(\theta_1) + P_{0,6y} \cdot cos(\theta_1)
    \label{eq:inv2-2}
\end{equation}
\begin{equation}
    \theta_5 = \pm \arccos \left( \frac{ P_{0,6x} \cdot \sin\theta_1 - P_{0,6y} \cdot \cos\theta_1 - d_4 }{ d_6 } \right)
    \label{eq:inv2-3}
\end{equation}
\begin{figure}[h]
    \centering
    \includegraphics[width = .5\textwidth]{Bilder/inv2}
    \caption{Berechnung von $\theta_5$, Betrachtung von allen Gelenken~\cite{rasmusandersenKinematicsUR52018}}\label{fig:inv1-2}
\end{figure}

\subsubsection{3. Gelenk sechs (Endeffektor)}

Nach $\theta_1$ und $\theta_5$ wird $\theta_6$ bestimmt.
Dazu wird die Eigenschaft des UR5 herangezogen, die Z-Achsen von Gelenken zwei, drei und vier liegt stets parallel zur Y-Achse von Gelenk 1 stehen (siehe Abbildung~\ref{fig:ur5-axis}).
Deshalb kann die Y-Achse $y_1$ beschrieben von Gelenk sechs ($Y_{6,1}$) unabhängig von $\theta_{1,2,3,4}$ und kann mithilfe von sphärischen Koordinaten als $-Y_{6,1}(-\theta_6,\theta_5)$ ($-\theta_5$ als Azimuth sowie $\theta_6$ als polaren Winkel) aufgefasst werden (Abbildung~\ref{fig:inv1-3}).
Eine Umrechnung in kartesische Koordinaten ergibt deshalb Gleichung~\ref{eq:inv3-1}, wobei $Y_{6,1}$ mithilfe einer Drehung um $\theta_1$ um $z_1$ beschrieben werden kann (Gleichung~\ref{eq:inv3-2}).
$Z$ und $X$ sind hierbei jeweils Einheitsvektoren der Z- und X-Achse des jeweiligen Koordinatensystems.
Nach Gleichsetzen der x- und y-Einträge der beiden Gleichungen und anschließendem Umformen erhält man $\theta_6$ (Gleichung~\ref{eq:inv3-3}).
Dabei kann es genau eine Lösung geben.
Falls $\sin\theta_5=0$, liegt eine Singularität vor und eine Lösung kann nicht bestimmt werden.
Dies tritt auf, wenn neben den Achsen $z_2$, $z_3$ und $z_4$ auch Achse $z_6$ parallel steht.
\begin{figure}[h]
    \centering
    \includegraphics[width = .4\textwidth]{Bilder/inv3}
    \caption{Berechnung von $\theta_6$ mit Sphärischen Koordinaten, für Koordinatensystembeschreibung siehe Abbildung~\ref{fig:ur5-axis}}\label{fig:inv1-3}
\end{figure}
\begin{equation}
    Y_{6,1}=
    \begin{bmatrix}
        -\sin\theta_5 \cdot \cos\theta_6 \\ \sin\theta_5 \cdot\sin\theta_6 \\ -\cos\theta_5
    \end{bmatrix}
    \label{eq:inv3-1}
\end{equation}
\begin{equation}
    Y_{6,1}=
    -\sin\theta_1\cdot X_{6,0} + \cos\theta_1\cdot Y_{6,0}
    \label{eq:inv3-2}
\end{equation}
\begin{equation}
    \theta_6=
    \arctantwo\left(
    \frac{
        -X_{6,0y} \cdot \sin\theta_1 + Y_{6,0y} \cdot \cos\theta_1
    }{
        \sin\theta_5
    },
    \frac{
        X_{6,0x} \cdot \sin\theta_1 - Y_{6,0x} \cdot \cos\theta_1
    }{
        \sin\theta_5
    }\right)
    \label{eq:inv3-3}
\end{equation}

\subsubsection{4. Gelenk drei (Ellenbogen)}

\subsubsection{5. Gelenk zwei (Schulter)}

\subsubsection{6. Gelenk vier (unteres Handgelenk)}

\subsubsection{Zusammenfassung}

\begin{equation}
    \overrightarrow{\theta} =
    \begin{bmatrix}
        \arctantwo(P_{0,5y}, P_{0}) \pm \arccos \left( \frac{d_4}{ \sqrt{ P_{0,5x}^2 + P_{0,5y}^2 }  } \right) + \frac{\pi}{2} \\
        xx                                                                                                                   \\
        xx                                                                                                                   \\
        xx                                                                                                                   \\
        \pm \arccos \left( \frac{ P_{0,6x} \cdot \sin\theta_1 - P_{0,6y} \cdot \cos\theta_1 - d_4 }{ d_6 } \right) \\
        \arctantwo\left(
        \frac{
            -X_{6,0y} \cdot \sin\theta_1 + Y_{6,0y} \cdot \cos\theta_1
        }{
            \sin\theta_5
        },
        \frac{
            X_{6,0x} \cdot \sin\theta_1 - Y_{6,0x} \cdot \cos\theta_1
        }{
            \sin\theta_5
        }\right)
    \end{bmatrix}\label{eq:inv-zusammenfassung}
\end{equation}


\section{Singularitäten}\label{sec:singularitaten}


Kuka vs UR5.

Theorie (rundungsfehler), Praxis (große Geschwindigkeiten)
alpha-2
alpha-1
alpha-5
Elbow-Up / Elbow-Down


\section{Geschwindigkeitskinematik}
    \cleardoublepage


\chapter{Pfadplanung}\label{ch:pfadplanung}

Bei der Pfadplanung geht es darum, eine Reihe an Konfigurationen zu berechnen, die den Weg von einer bestimmten Transformation in eine andere beschreiben.
Eine einzelne Konfiguration ist im Falle des UR5 ein Tupel mit sechs Winkeln, die die Ausrichtung der sechs Gelenke des Roboters beschreiben.

Die einfachste Form der Pfadplanung ist der kürzeste Weg zwischen zwei Punkten (Abschnitt~\ref{subsec:kurzester-weg}), die schnellste Verbindung ist immer der Weg der direkten Kinematik mit maximaler Winkelgeschwindigkeit bei der Bewegung (Abschnitt~\ref{ch:direkte-kinematik}).
Komplizierter wird es, wenn Beschränkungen im Arbeitsraum des Roboters, wie Kollisionen mit der Umgebung oder mit sich selbst, vorliegen.
In diesem Fall muss ein komplizierteres Berechnungsverfahren herangezogen werden.

In diesem Kapitel sollen die Grundlagen der Pfadplanung, sowie einige Algorithmen näher vorgestellt werden.


\section{Konfigurationsraum}\label{sec:konfigurationsraum}


Der Konfigurationsraum $\mathit{C}$ stellt die Grundlage für die Algorithmen in der Pfadplanung dar und beschreibt alle möglichen Konfigurationen, die der Roboter einnehmen kann.
Eine Konfiguration des UR5 ist dabei wie in Gleichung~\ref{eq:config-1} definiert, wobei die Winkelwerte $\theta_i$ durch ihre Maximal- und Minimalwerte des Datenblatts~\cite{universalrobotsUR5TechnicalSpecifications} von $\pm 360^{\circ}$ beschränkt sind.
Gleichsam existieren Konfigurationen innerhalb dieser Menge $\mathit{O}\in\mathit{C}$, die Kollisionen beschreiben.
Darunter zählen Konfigurationen, bei denen sich der Roboter selbst im Weg steht, sowie welche, bei denen der Untergrund oder statische Hindernisse eine erfolgreiche Ausführung der Bewegung verhindern.
Der für die Pfadplanung relevante Bereich ist demzufolge der in Gleichung~\ref{eq:config-2} beschriebene Menge $\mathit{C}_{free}$, der freie Konfigurationsraum, dessen Konfigurationen den Bereich beschreiben, durch den Roboter sich bewegen kann.

\begin{equation}
    \mathit{C} = \left\{ \left( \theta_1,\dots,\theta_6 \right) \in \theta_1\times\dots\times\theta_6 \mid \theta_i \in \left[ -2\pi, 2\pi \right]\right\}
    \label{eq:config-1}
\end{equation}
\begin{equation}
    \mathit{C}_{free} = \mathit{C}\backslash\mathit{O}
    \label{eq:config-2}
\end{equation}

Je nach Algorithmus muss der Konfigurationsraum allerdings nicht im vorhinein berechnet werden.
Um den Kollisionsraum abzubilden wurde der Roboter in dieser Arbeit mit Kugeln und Zylindern in den Dimensionen des UR5 approximiert und in fünf-Grad Schritten jede Position auf Eigenkollision getestet, damit Pfadplanungsalgorithmen schneller auf ihre Funktionsweise hin untersucht werden können.

?? Code

??Quelle

% todo berechnung/beschreibung von collision detection mit bb / trees
https://www.geometrictools.com/Documentation/DynamicCollisionDetection.pdf


\section{Berechnungsmethoden}
%
%\subsection{Kürzester Weg}\label{subsec:kurzester-weg}
Im einfachsten Fall soll der Roboter auf kürzestem Weg in eine Zielposition fahren.
Dabei wird die Wegstrecke meist linear in mehrere Zwischentransformationen zwischen zwei oder mehr Transformationen $T_1$ und $T_2$ aufgeteilt.
Für die Translation von Punkt $P_1$ und Punkt $P_2$ kann für jeden Schritt ein Faktor $t \in \left[0,1\right]$ multipliziert werden (Gleichung~\ref{eq:shortest-path-1}).
Zwischen zwei Rotationen $R_1$ und $R_2$ wird in der Regel der sog.\ Slerp verwendet (Gleichung~\ref{eq:shortest-path-2}).
Dabei müssen die Rotationen der allerdings erst in Quaternionen $Q_1$ und $Q_2$ umgerechnet werden.
Zu beachten sind zudem die Rechenregeln der Quaternionen, die hier nicht näher erklärt werden.

\begin{equation}
    P_{t} = P_0 + \left( P_1 - P_0 \right) \cdot t
    \label{eq:shortest-path-1}
\end{equation}
\begin{equation}
    Q_{t} = Q_1 \cdot \left( Q_1^{-1} \cdot Q_2 \right)^t
    \label{eq:shortest-path-2}
\end{equation}

Die resultierende Liste an Endeffektor-Transformationen muss dann in den Konfigurationsraum des Roboters übersetzt werden.
Beim UR5 gibt es, wie in Abschnitt~\ref{sec:geometrische-losung} berechnet, für jede Zwischentransformation oftmals bis zu acht, oder in einer Singularität sogar unendlich viele Stellungen für die Gelenke.
Um einen geeigneten Pfad durch den Konfigurationsraum zu nehmen und die Selbstkollision sowie das Erreichen von Winkelbegrenzungen zu vermeiden, sollte deshalb ein Pfadplanungsalgorithmus zu Hilfe genommen werden.

?? Quelle

\subsection{Zellendekomposition}\label{subsec:zellendekomposition}
?? Quelle

In der Zellendekomposition wird der freie Konfigurationsraum $C_{free}$ in kleinere Felder, sog.\ Zellen unterteilt.
Zwei Konfigurationen liegen genau dann in der gleichen Zelle, falls ein Übergang zwischen den beiden Zuständen kollisionsfrei möglich ist und zwei benachbarte Zellen müssen über einen einfachen Pfad miteinander kollisionsfrei verbunden sein.
Dies setzt konvexe Zellen voraus, bei denen jeder Punkt innerhalb der Zelle jeden anderen erreichen kann.
Auf Basis dieser Gruppierungen kann dann ein Konnektivitätsgraph oder eine Roadmap erstellt werden.
Als Datenstruktur kann ein KD-Tree verwendet werden, der ein Konfigurationsraum als Baum so abbildet, dass Abstände und benachbarte Zellen zügig berechnet werden können.
Im zweidimensionalen Raum werden hierfür die Positionen und Dimensionen von Quadraten sowie dessen Nachbarn gespeichert.

Diese direkte Berechnung der Zellenstruktur ist im sechsdimensionalen Fall des UR5 allerdings nichttrival und aufwändig.
Aus diesem Grund und für die schnellere Berechnung wird deshalb oftmals eine approximierte Lösung aus einem Sampling-Verfahren bevorzugt.

\subsection{Sampling-Verfahren}
?? Quelle

In Sampling-basierten Verfahren wird das Gruppieren aller Konfigurationen in konkave Zellen übersprungen und stattdessen nur einzelne, benötigte Konfigurationen auf Kollision geprüft.
Das bedeutet, dass nicht, wie in Abschnitt~\ref{subsec:zellendekomposition} alle Lösungen für alle Probleme generalisiert werden sollen, sondern eine einzelne Lösung für ein einzelnes Problem gesucht ist.
Bedingung dafür ist allerdings eine Möglichkeit, eine Konfiguration schnell auf Kollisionsfreiheit zu überprüfen.
Insofern der Konfigurationsraum wie in Abschnitt~\ref{sec:konfigurationsraum} schon zuvor berechnet wurde, kann alternativ ein Lookup-Table verwendet werden.
Dabei kann zwischen Single- und Multi-Query-Verfahren unterschieden werden.

\subsubsection{Single-Query}

Bei Single-Query-Verfahren wird für jede Abfrage ein neuer Weg gesucht, bei Multi-Query-Verfahren werden vergangene Anfragen als Hilfestellung für neue Anfragen genutzt und so eine Roadmap erzeugt.
Üblicherweise wird beim Single-Query-Verfahren mit Startkonfiguration $q_a$ und Zielkonfiguration $q_b$ wie folgt vorgegangen:
\begin{enumerate}
    \item Initalisierung eines Graphens $G(V, E)$ mit Knoten mit $V=\left\{ q_a, q_b \right\}$ und $E=\left\{  \right\}$
    \item Auswahl eines Expansionsknotens aus $V$ mit einem gewählten Algorithmus
    \item Berechnung eines neuen Knotens und einer Kante mit einem gewählten Algorithmus (inklusive Kollisionsüberprüfung)
    \item Erweiterung des Graphens mit zuvor gefundenem Knoten und neuer Kante (falls möglich).
    \item Prüfe auf Lösung in G (Verbindung von Start und Zielknoten) oder wiederhole ab 2.
\end{enumerate}

Zudem kann der Algorithmus auf zwei oder mehr Suchbäume ausgeweitet werden, die sich nach einiger Zeit in der Mitte treffen.
Dazu wird dann in den Schritten 2\-4 der verwendete Baum zufällig ausgewählt.

Der \ac{rrt} ist ein häufig verwendeter Suchbaum in diesem Kontext.
Hier wird zunächst wie in 3.\ eine neue Konfiguration zufällig gewählt und ab dem nächsten Knoten im Graphen eine Schrittweite in Richtung dieser Konfiguration gegangen.
Um einen Bias gegenüber der Zielkonfiguration zu erzeugen, kann zudem mit einer gewissen Wahrscheinlichkeit der zufällig gewählte Knoten mit dem tatsächlichen Zielknoten ersetzt werden.
Bei RRT$^*$ wird bei der Erweiterung zusätzlich auf andere Verbindungen des Graphens zum neu berechneten Knoten geprüft, was die Länge der kürzesten Verbindung zwischen Start und Ziel noch einmal minimiert.
Die kürzeste Verbindung kann beispielsweise mithilfe von A$^*$ oder Dijkstra ermittelt werden.

\subsubsection{Multi-Query}
Bei Multi-Query Verfahren können bereits berechnete Kanten des aufgebauten Graphens wiederverwendet werden.
Im Probabilistic-Roadmap-Algorithmus werden in einem ersten Schritt n zufällige Knotenpunkte erzeugt und dann mit ihren nächsten Nachbarn verbunden, insofern keine Kollision vorliegt.
Im Anschluss werden Start- und Zielknoten ergänzt und ebenfalls Nachbarn als Kanten im Graphen hinzugefügt.
Am Ende kann dann wieder mithilfe von A$^*$ oder Dijkstra ein kürzester Weg ermittelt werden.


??

%Single-Query
%Unidirektional vs.\ Bidirektional
%RRT
%biased / unbiased to exploration
%Q-space muss nicht vollständig bekannt sein
%RRT*
%Multi-Query
%Probabilistic Roadmaps
Potentialfeldmethode / Gradientenverfahren
Genetische Algorithmen


\section{Constraints und Praxisbezug}
Kinematisch (Winkelbegrenzung)
Dynamisch (Geschwindigkeit)
Einbezug der Constraints in den Algorithmen
Praxis: OMPL (MoveIt+ ROS / Copelliasim Plugin)


    \cleardoublepage


\chapter{Trajektorienplanung}


\section{Profile}

Trapez, 7-Segment


\section{Synchronität}

Vollsynchron, Teilsynchron, Asynchron


\section{Mehrsegment-Trajektorien}

Ggf besser in Pfadplanung ??

Mehrsegment-Trajektorien (z.B. Bezier, Überschleifen)

    \cleardoublepage
\chapter{Fazit und Ausblick}

%    \input{006_Konzept.tex}
%    \input{007_Umsetzung.tex}
%    \input{008_Verwendung.tex}
%    \input{009_Ausblick.tex}
%%%%%%%%%%%%%%%%%%%%%%%%%%%%%%%%%%%%%%%%%%%%%%%%%%%%%%%%%%%%%%%%%%%%%%%%%
    \cleardoublepage
    \printbibliography    % Lit.verzeichnis
%    \input{198_Anhang.tex}
    \input{199_Abschlusserklaerung.tex}
%%%%%%%%%%%%%%%%%%%%%%%%%%%%%%%%%%%%%%%%%%%%%%%%%%%%%%%%%%%%%%%%%%%%%%%%%
\end{document}
%%%%%%%%%%%%%%%%%%%%%%%%%%%%%%%%%%%%%%%%%%%%%%%%%%%%%%%%%%%%%%%%%%%%%%%%%
