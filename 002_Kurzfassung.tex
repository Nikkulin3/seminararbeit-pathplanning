\chapter*{Kurzfassung}
\addcontentsline{toc}{chapter}{Kurzfassung}
\chaptermark{Kurzfassung}


In dieser Arbeit wird die Steuerung des UR5 näher untersucht und eine Ursache der häufig auftretenden Selbstkollisionen ermittelt.
Zudem werden alternative Algorithmen erläutert und die Vor- und Nachteile der verschiedenen Methoden verglichen.
Dabei wird eine direkte und schnelle Berechnungsmethode zur Bestimmung der Gelenkwinkel vorgestellt und die auftretenden Selbstkollisionen durch dadurch begründet, dass die Software des UR5 vermutlich stattdessen auf einer numerischen Methode der inversen Kinematik basiert.
Es wird festgestellt, dass für jede Endeffektor-Transformation in der Regel bis zu acht mögliche Roboterstellungen existieren und ein Algorithmus vorgestellt,
der ohne komplizierte Pfadplanung alle Lösungswege für eine gerade Bahn berechnen und auf Kollision überprüfen kann.