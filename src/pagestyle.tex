% Wird auf alle nicht leeren Seiten angewandt
% siehe auch;
% http://tug.ctan.org/tex-archive/macros/latex/contrib/fancyhdr/fancyhdr.pdf
\fancypagestyle{title}{%
	\fancyhf{}
	\fancyhead[L]{Fakultät für Angewandte Informatik \\
		\textbf{Lehrstuhl für Produktionsinformatik} \\ Prof. Dr.-Ing. Johannes Schilp}

	\fancyhead[R]{
		\raisebox{0pt}[0pt][0pt]{\includegraphics[width=3.5cm]{src/Uni_Aug_Logo_Basis_pos_A.pdf}}
	}

	\renewcommand{\headrulewidth}{0.4pt} %obere Trennlinie
}


\renewcommand{\chaptermark}[1]{\markboth{
		\thechapter. #1}{}}
%\renewcommand{\sectionmark}[1]{\markboth{
%		\thesection. #1}{}}


\pagestyle{fancyplain}% <- pagestyle fancyplain
\renewcommand\plainheadrulewidth{.4pt}% headrule on plain pages
\fancyhf{}
\fancyhead[L]{\namedesautors}
\fancyhead[R]{\leftmark}
%\fancyhead[R]{\rightmark}
\fancyfoot[RE,LO]{}
\fancyfoot[LE,RO]{\pagemark}
\renewcommand{\chaptermark}[1]{%
	\markboth{#1}{}}


\headsep = 20pt % COOLER CODE: Das Ding setzt die Margin zwischen Header und Text darunter
				% Bei Überschriften setzt sich diese Margin aber aus dieser bottom Margin + der
				% top Margin von Überschriften zusammen. Diese kann mit dem folgenden Command
				% geändert werden:
\renewcommand*{\chapterheadstartvskip}{\vspace*{0cm}} % COOLER CODE:
													  % Setzt top Margin bei \chapter. Wenn dus
													  % so haben willst wie vorher, einfach die
													  % Zeile löschen. Selbe gilt bei \headsep

%%  Der Vollständigkeithalber die alten verschiedenen Pagestyles von Peter
%   pagestyle.tex
%   Version 1.0     |   Peter Krönes    |   08.05.2018

%%%%%%%%%%%%%%%%%%%%%%%  PAGESTYLE  %%%%%%%%%%%%%%%%%%%%%%%%%%%%%%%%%%%%
%
%\usepackage{fancyhdr}
%\pagestyle{fancy}
%\renewcommand{\chaptermark}[1]{\markboth{
%		\thechapter. #1}{}}
%\renewcommand{\sectionmark}[1]{\markboth{
%		\thesection. #1}{}}
%
%%%%%%%%%%%%%%%%%%%%%%%%%%%%%%%%%%%%%%%%%%%%%%%%%%%%%%%%%%%%%%%%%%%%%%%%
%\fancypagestyle{scrheadings}{
%	\fancyhead[LE,RO]{\nouppercase\leftmark} 	% Kapitelname oben ins andere Eck
%	\fancyhead[LO,RE]{\namedesautors}	% Autor oben im Eck
%	\fancyfoot[LE,RO]{\pagemark} 		% Seitenzahl unten im Eck
%	\fancyfoot[C]{}
%}
%%%%%%%%%%%%%%%%%%%%%%%%%%%%%%%%%%%%%%%%%%%%%%%%%%%%%%%%%%%%%%%%%%%%%%%%
% Pagestyle für die erste Seite der List of Figures/Tables
%\fancypagestyle{pagenumberstyle}{
%	\renewcommand{\headrulewidth}{0pt}
%
%	\fancyhead[LE,RO]{\chaptermark} 	% Kapitelname oben ins andere Eck
%	\fancyhead[LO,RE]{\namedesautors}	% Autor oben im Eck
%	\fancyfoot[LE,RO]{\pagemark} 		% Seitenzahl unten im Eck
%	\fancyfoot[C]{}
%}
% Unterscheidung zwischen einseitig und zweiseitigem Druck nicht nötig, da Kapitel bei zweiseitigem Druck immer rechts anfangen.
%%%%%%%%%%%%%%%%%%%%%%%%%%%%%%%%%%%%%%%%%%%%%%%%%%%%%%%%%%%%%%%%%%%%%%%%
%
%%%%%%%%%%%%%%%%%%%%%%%%%%%%%%%%%%%%%%%%%%%%%%%%%%%%%%%%%%%%%%%%%%%%%%%%
% Pagestyle für scrartcl
%\fancypagestyle{articlestyle}{
%	\renewcommand{\headrulewidth}{0.1pt}
%	\fancyhead[LE,RO]{\chaptermark} 	% Kapitelname oben ins andere Eck
%	\fancyhead[LO,RE]{\namedesautors}	% Autor oben im Eck
%	\fancyfoot[LE,RO]{\pagemark} 		% Seitenzahl unten im Eck
%	\fancyfoot[C]{}
%}
% Unterscheidung zwischen einseitig und zweiseitigem Druck nicht nötig, da Kapitel bei zweiseitigem Druck immer rechts anfangen.
%%%%%%%%%%%%%%%%%%%%%%%%%%%%%%%%%%%%%%%%%%%%%%%%%%%%%%%%%%%%%%%%%%%%%%%%



%% wurde dann vor listoffigures, listoftables, tableofcontents und printbibliography
%% wie folgt eingestellt: \tableofcontents{\thispagestyle{pagenumberstyle}}
