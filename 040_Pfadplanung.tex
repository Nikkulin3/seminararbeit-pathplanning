\cleardoublepage


\chapter{Pfadplanung}\label{ch:pfadplanung}

Bei der Pfadplanung geht es darum, eine Reihe an Konfigurationen zu berechnen, die den Weg von einer bestimmten Transformation in eine andere beschreiben.
Eine einzelne Konfiguration ist im Falle des UR5 ein Tupel mit sechs Winkeln, die die Ausrichtung der sechs Gelenke des Roboters beschreiben.

Die einfachste Form der Pfadplanung ist der kürzeste Weg zwischen zwei Punkten (Abschnitt~\ref{subsec:kurzester-weg}), die schnellste Verbindung ist immer der Weg der direkten Kinematik mit maximaler Winkelgeschwindigkeit bei der Bewegung (Abschnitt~\ref{ch:direkte-kinematik}).
Komplizierter wird es, wenn Beschränkungen im Arbeitsraum des Roboters, wie Kollisionen mit der Umgebung oder mit sich selbst, vorliegen.
In diesem Fall muss ein komplizierteres Berechnungsverfahren herangezogen werden.

In diesem Kapitel sollen die Grundlagen der Pfadplanung, sowie einige Algorithmen näher vorgestellt werden.


\section{Konfigurationsraum}\label{sec:konfigurationsraum}


Der Konfigurationsraum $\mathit{C}$ stellt die Grundlage für die Algorithmen in der Pfadplanung dar und beschreibt alle möglichen Konfigurationen, die der Roboter einnehmen kann.
Eine Konfiguration des UR5 ist dabei wie in Gleichung~\ref{eq:config-1} definiert, wobei die Winkelwerte $\theta_i$ durch ihre Maximal- und Minimalwerte des Datenblatts~\cite{universalrobotsUR5TechnicalSpecifications} von $\pm 360^{\circ}$ beschränkt sind.
Gleichsam existieren Konfigurationen innerhalb dieser Menge $\mathit{O}\in\mathit{C}$, die Kollisionen beschreiben.
Darunter zählen Konfigurationen, bei denen sich der Roboter selbst im Weg steht, sowie welche, bei denen der Untergrund oder statische Hindernisse eine erfolgreiche Ausführung der Bewegung verhindern.
Der für die Pfadplanung relevante Bereich ist demzufolge der in Gleichung~\ref{eq:config-2} beschriebene Menge $\mathit{C}_{free}$, der freie Konfigurationsraum, dessen Konfigurationen den Bereich beschreiben, durch den Roboter sich bewegen kann.

\begin{equation}
    \mathit{C} = \left\{ \left( \theta_1,\dots,\theta_6 \right) \in \theta_1\times\dots\times\theta_6 \mid \theta_i \in \left[ -2\pi, 2\pi \right]\right\}
    \label{eq:config-1}
\end{equation}
\begin{equation}
    \mathit{C}_{free} = \mathit{C}\backslash\mathit{O}
    \label{eq:config-2}
\end{equation}

Je nach Algorithmus muss der Konfigurationsraum allerdings nicht im vorhinein berechnet werden.

??Quelle


\section{Berechnungsmethoden}

\subsection{Kürzester Weg}\label{subsec:kurzester-weg}

\subsection{Zellendekomposition}
Zellendekomposition (Vollständige Abtastung)

\subsection{Sampling-Verfahren}
Single-Query
Unidirektional vs. Bidirektional
RRT
biased / unbiased to exploration
Q-space muss nicht vollständig bekannt sein
RRT*
Multi-Query
Probabilistic Roadmaps
Potentialfeldmethode / Gradientenverfahren
Genetische Algorithmen


\section{Constraints und Praxisbezug}
Kinematisch (Winkelbegrenzung)
Dynamisch (Geschwindigkeit)
Einbezug der Constraints in den Algorithmen
Praxis: OMPL (MoveIt+ ROS / Copelliasim Plugin)

