\chapter*{Aufgabenstellung}
\addcontentsline{toc}{chapter}{Aufgabenstellung}
\chaptermark{Aufgabenstellung}

In der Industrie der heutigen Zeit werden zur Herstellung verschiedenster Produkte Industrieroboter eingesetzt.
Die Aufgabe der Forschung ist es hierbei, die Fähigkeiten dieser Roboter auf neue Produktionsverfahren zu übertragen.
In der Universität wird der Roboter UR5 von Universal Robotics unter anderem verwendet, um automatisierte additive Fertigung zu ermöglichen.
Dabei sollen die gefertigten Bauteile automatisiert aus dem Druckbereich entfernt und an anderer Stelle platziert werden.

In dieser Arbeit soll die Steuerung des UR5 näher untersucht werden, die einige Probleme aufweist.
Es wurde unter anderem festgestellt, dass der Roboter beim Fahren zwischen zwei Positionen oftmals in eine Position fährt, in der es nicht mehr weitergeht und in Folge ein Notstopp durchgeführt werden muss.
Es gilt, eine mögliche Ursache für dieses Problem zu identifizieren und eine Lösung zu präsentieren, die es ermöglicht Kollisionen frühzeitig zu erkennen und bei möglichst gleichem Pfad durch andere Bewegungsmuster zu umgehen.