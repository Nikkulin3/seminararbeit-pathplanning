\chapter*{Aufgabenstellung}
\addcontentsline{toc}{chapter}{Aufgabenstellung}
\chaptermark{Aufgabenstellung}

In der Industrie der heutigen Zeit werden zur Herstellung verschiedenster Produkte Industrieroboter eingesetzt.
Die Aufgabe der Forschung ist es daher, die eingesetzten Verfahren zu optimieren und die Verwendung neuer Produktionsmethoden mithilfe dieser Roboter zu optimieren.
In der Universität wird der Roboter UR5 von Universal Robotics unter anderem eingesetzt, um automatisierte additive Fertigung zu ermöglichen.
Dabei sollen die gefertigten Bauteile automatisiert aus dem Druckbereich entfernt und an anderer Stelle platziert werden.

In dieser Arbeit soll die Steuerung des UR5 näher untersucht werden, die einige Probleme aufweist.
Es wurde unter anderem festgestellt, dass beim Fahren zwischen zwei Positionen oftmals eine Selbstkollision auftritt und in Folge ein Notstopp durchgeführt werden muss.
Es gilt, eine mögliche Ursache für dieses Problem zu identifizieren und eine Lösung zu präsentieren, die es ermöglicht Selbstkollision zu erkennen und bei gleichem Pfad zu umgehen.